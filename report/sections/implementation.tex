\section{Design pattern}

\subsection{Singleton}

\subsubsection{Fichier de configuration}

\subsection{Factory}
\paragraph{}Nous nous sommes basés sur un système de factory pour l'instanciation de
certaines classes. Cela nous permet d'instancier une certaine factory et après
de pouvoir gérer l'instanciation par exemple des agents de façon
générique. 

\section{Agent}


\subsection{Comportements}
\paragraph{}Le but était d'implementer deux types de comportement :
constructeur et destructeur. Le
comportement de type constructeur (ogre) doit permettre de rassembler les objets
en endroit du terrain. Le comportment de type destructeur (robot) a pour
objectif de disperser les objets sur le terrain.

La classe \texttt{Behavior} regroupe les méthodes de bases. On trouve des
méthodes pour effectuer un déplacement vers une position visée, pour choisir
une nouvelle destination. La méthode \texttt{Update} sera appelée à chaque frame.

Afin de rassembler le code en commun dans les comportements constructeurs et
destructeurs, une classe \texttt{CarriedBehavior} a été faites. Cette classe
hérite donc de la classe \texttt{Behavior}.  Elle rassemble les mécanisme pour
ramasser des objets et les déposer. Nos comportements constructeurs et destructeurs 
hérite de cette classe.


\subsubsection{Builder}
\paragraph{}Pour les comportement de type constructeur deux types de comportement ont été
réalisés. Le premier comportement \texttt{BuilderBehavior} peut être assimiler à un petit automate à 4 états.


\begin{description}
    \item[\'Etat 1] l'ogre ne porte pas d'objet. Si un objet est à porté il
        le ramasse et passe dans l'état 2.
    \item[\'Etat 2] l'ogre vient de ramasser un objet. Il se déplace pendant une
        durée aléatoire sans faire d'autres actions puis passe dans l'état 3.
    \item[\'Etat 3] l'ogre porte un objet. Si un objet est à porté il
        dépose l'objet qu'il porte à côté de celui ci. Il passe ensuite dans
        l'état 4.
    \item[\'Etat 4] l'ogre vient de poser un objet. Il se déplace pendant une
        durée aléatoire sans faire d'autres actions puis passe dans l'état 1.
\end{description}

\paragraph{}Gràce à ce comportement les ogres sont capables de rassembler les caisses
petit à petit. On voir rapidement émerger plusieurs tas. Cependant la
convergence vers un seul tas unique peut être particuliérement longue.

\paragraph{}Le deuxiéme comportement \texttt{CleverBuilderBahavior} est quand à lui plus
complexe. On ajoute une mémoire aux ogres : ils se rappellent l'endroit où ils ont vu le
plus de caisses. Et ainsi, quand ils désirent poser leur objet alors ils se rendent à
cet endroit.

\begin{description}
    \item[\'Etat 1] l'ogre ne porte pas d'objet. Si un objet est à porté il
        le ramasse et passe dans l'état 2.
    \item[\'Etat 2] l'ogre vient de ramasser un objet. Il se déplace pendant une
        durée aléatoire en regardant le nombre d'objets qui sont dans son
        champ de vision et en mettant à jour sa mémoire si il découvre un
        endroit avec un nombre d'objet supérieure. Puis il passe dans l'état 3.
    \item[\'Etat 3] l'ogre porte un objet. L'ogre se rend à l'endroit
        enregistré dans sa mémoire et y dépose l'objet. Il profite pour mettre
        le nombre d'objet s'y trouvant à jour par la même occasion. Il passe
        ensuite dans l'état 4.
    \item[\'Etat 4] l'ogre vient de poser un objet. Il se déplace pendant une
        durée aléatoire sans faire d'autres actions puis passe dans l'état 1.
\end{description}

\paragraph{}Ce comportement permet d'augmenter de façon important l'éfficacité
des ogres. Au bout d'une durée relativement courte les ogres ont tous trouvé
l'endroit comportent le plus de caisses. 

\subsubsection{Wrecker}
\paragraph{}Au niveau du comportement de type destructeur, une seule version a
été implementé : \texttt{WrekerBehavior}. Celle ci ressemble fortement au premier comportement des
ogres mais les conditions pour poser un objet sont différentes. 

\begin{description}
    \item[\'Etat 1] l'ogre ne porte pas d'objet. Si un objet est à porté il
        le ramasse et passe dans l'état 2.
    \item[\'Etat 2] l'ogre vient de ramasser un objet. Il se déplace pendant une
        durée aléatoire sans faire d'autres actions puis passe dans l'état 3.
    \item[\'Etat 3] l'ogre porte un objet. Si aucun objet est à porté il
        dépose l'objet qu'il porte. Il passe ensuite dans
        l'état 4.
    \item[\'Etat 4] l'ogre vient de poser un objet. Il se déplace pendant une
        durée aléatoire sans faire d'autres actions puis passe dans l'état 1.
\end{description}

\paragraph{}Se comportement permet au robot de disséminer les caisses.
Cependant, on remarque que l'efficacité de ce comportement augmente quand les
caisses ne sont pas réparties de façon homogéne sur l'ensemble du terrain. En
effet, dans ce cas là les robots peuvent alors mettre en temps important à
trouver un endroit où ils n'ont pas d'autres caisses en vue. 

\section{Modélisation 3D}


\subsection{Techniques mises en place}
\subsubsection{SkyBox}
Nous avons essayé plusieurs méthodes pour le ciel : SkyBox et SkyDome. Le
skydome n'était pas adapté à notre application. Notre terrain étant une
surface carré de dimenssion fini il était possible voir le bas du skydome.
Hors comme le skydome consiste à appliquer une image carré sur un dome, le
rendu sur le bas du dome est déformé. Il est donc préférable d'utiliser une
skybox.

\subsubsection{Mesh}
Pour réaliser nos caisses, nous avons utiliser le logiciel de modélisation 3D
blender afin de réaliser le mesh. Le mesh correspond tout simplement à un
cube. Nous avons ensuite fais une texture basique avec gimp que nous avons
appliquer sur notre cube en utilisant la méthode "UV texture map".
Nous avons ensuite exporté le mesh afin de le mettre dans le format
accepter par le moteur OGRE 3D. 

% Fog ?
% Lights ?
% Overlay

\subsection{Interaction avec l'utilisateur}

% Mesh selection ?
% -> put a light on an agent
% -> kill an agent
% -> move an agent
% Overlay
% -> help (key bindings)
% -> informations (nb of each agent/death/birth, life speed, time from start)
% -> debugging ?
% Life speed
% Camera movement
% Pause
% Ambient light

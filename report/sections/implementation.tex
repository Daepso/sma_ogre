\section{Agent}

\subsection{Comportements}
\paragraph{}Le but était d'implémenter deux types de comportement :
constructeur et destructeur. Le
comportement de type constructeur (ogre) doit permettre de rassembler les objets
en endroit du terrain. Le comportement de type destructeur (robot) a pour
objectif de disperser les objets sur le terrain.

La classe \texttt{Behavior} regroupe les méthodes de bases. On trouve des
méthodes pour effectuer un déplacement vers une position visée, pour choisir
une nouvelle destination. La méthode \texttt{Update} sera appelée à chaque frame.

\paragraph{}Afin de rassembler le code en commun dans les comportements constructeurs et
destructeurs, une classe \texttt{CarrierBehavior} a été créée. Cette classe
hérite donc de la classe \texttt{Behavior}.  Elle rassemble les mécanisme pour
ramasser des objets et les déposer. Nos comportements constructeurs et destructeurs
hérite de cette classe.

\paragraph{}Afin que les agents n'est qu'une vision local du système, une
distance de vue a été ajoutée. Grâce à cela, quand les agents récupère la
liste des objets sur le terrain, seulement les objets dans leur champ de
vision leur sont envoyés. Il ne peuvent alors raisonner que sur une partie des
objets.

Une distance de d'interaction a aussi été mis en place. Les agents ont besoin
d'être à une certaine distance d'un objet pour pouvoir le ramasser.

\subsubsection{Builder}
\paragraph{}Pour les comportement de type constructeur deux types de comportement ont été
réalisés. Le premier comportement \texttt{BuilderBehavior} peut être assimilé à un petit automate à 4 états.


\begin{description}
    \item[\'Etat 1] l'ogre ne porte pas d'objet. Si un objet est à porté il
        le ramasse et passe dans l'état 2. Sinon si au moins un objet est dans son
        champ de vision alors il se dirige vers l'objet le plus proche.
    \item[\'Etat 2] l'ogre vient de ramasser un objet. Il se déplace pendant une
        durée aléatoire sans faire d'autres actions puis passe dans l'état 3.
    \item[\'Etat 3] l'ogre porte un objet. Si un objet est à porté il
        dépose l'objet qu'il porte à côté de celui-ci. Il passe ensuite dans
        l'état 4.
    \item[\'Etat 4] l'ogre vient de poser un objet. Il se déplace pendant une
        durée aléatoire sans faire d'autres actions puis passe dans l'état 1.
\end{description}

\paragraph{}Grâce à ce comportement les ogres sont capables de rassembler les caisses
petit à petit. On voir rapidement émerger plusieurs tas. Cependant la
convergence vers un seul tas unique peut être particulièrement longue.

\paragraph{}Le deuxième comportement \texttt{CleverBuilderBahavior} est quand à lui plus
complexe. On ajoute une mémoire aux ogres : ils se rappellent l'endroit où ils ont vu le
plus de caisses. Et ainsi, quand ils désirent poser leur objet ils se rendent à
cet endroit.

\begin{description}
    \item[\'Etat 1] l'ogre ne porte pas d'objet. Si un objet est à porté il
        le ramasse et passe dans l'état 2. Sinon si au moins un objet est dans son
        champ de vision alors il se dirige vers l'objet le plus proche.
    \item[\'Etat 2] l'ogre vient de ramasser un objet. Il se déplace pendant une
        durée aléatoire en regardant le nombre d'objets qui sont dans son
        champ de vision et en mettant à jour sa mémoire si il découvre un
        endroit avec un nombre d'objet supérieure. Puis il passe dans l'état 3.
    \item[\'Etat 3] l'ogre porte un objet. L'ogre se rend à l'endroit
        enregistré dans sa mémoire et y dépose l'objet. Il profite pour mettre
        le nombre d'objet s'y trouvant à jour par la même occasion. Il passe
        ensuite dans l'état 4.
    \item[\'Etat 4] l'ogre vient de poser un objet. Il se déplace pendant une
        durée aléatoire sans faire d'autres actions puis passe dans l'état 1.
\end{description}

\paragraph{}Ce comportement permet d'augmenter de façon importante l'efficacité
des ogres. Au bout d'une durée relativement courte les ogres ont tous trouvé
l'endroit comportent le plus de caisses.

\paragraph{}
Cependant les ogres étant devenus trop efficaces par rapport aux robots, nous
avons ajouté un dernier comportement \texttt{MortalCleverBuilderBahavior} les
rendant mortels. Leur espérance de vie est basée sur un temps de demi-vie -
par analogie à la désintégration de corps radioactifs - au bout duquel environ
la moitié de la population va disparaître. Cette espérance de vie est une
propriété associée à ce nouveau comportement qui fait qu'à chaque mise à jour
du comportement (toutes les frame) l'agent a une probabilité de mourir
$P(death)$ valant :

\[ P(death) = \lambda dt \]

\[ où \left\{
        \begin{array}{l}
            \lambda  = \frac{\ln 2}{t_{1/2}}\\
            t_{1/2} \quad \verb!est le temps de demi-vie!\\
            dt \quad \verb!est le temps écoulé depuis la dernière mise à jour!
    \end{array} \right. \]

\subsubsection{Wrecker}
\paragraph{}Au niveau du comportement de type destructeur, une seule version a
été implémenté : \texttt{WrekerBehavior}. Celle-ci ressemble fortement au premier comportement des
ogres mais les conditions pour poser un objet sont différentes.

\begin{description}
    \item[\'Etat 1] l'ogre ne porte pas d'objet. Si un objet est à porté il
        le ramasse et passe dans l'état 2. Sinon si au moins un objet est dans son
        champ de vision alors il se dirige vers l'objet le plus proche.
    \item[\'Etat 2] l'ogre vient de ramasser un objet. Il se déplace pendant une
        durée aléatoire sans faire d'autres actions puis passe dans l'état 3.
    \item[\'Etat 3] l'ogre porte un objet. Si aucun objet est à porté il
        dépose l'objet qu'il porte. Il passe ensuite dans
        l'état 4.
    \item[\'Etat 4] l'ogre vient de poser un objet. Il se déplace pendant une
        durée aléatoire sans faire d'autres actions puis passe dans l'état 1.
\end{description}

\paragraph{}Ce comportement permet au robot de disséminer les caisses.
Cependant, on remarque que l'efficacité de ce comportement augmente quand les
caisses ne sont pas réparties de façon homogène sur l'ensemble du terrain. En
effet, dans ce cas là les robots peuvent alors mettre en temps important à
trouver un endroit où ils n'ont pas d'autres caisses en vue.

\section{Fonctionnalités}

Nous avons implémenté diverses fonctionnalités au cours de ce projet. Nous
avons classé ces fonctionnalités en trois types : les techniques de rendu 3D,
les overlays - qui permettent d'ajouter des éléments 2D en superposition sur
la scène - et les interactions avec l'utilisateur.

\subsection{Techniques de rendu 3D}

\subsubsection{Mode nuit}
Une des premières fonctionnalités que nous avons mises en place est une mode
nuit. Dans un premier temps nous nous sommes contentés de modifier la lumière
ambiante pour donner cette impression de jour et de nuit. Par la suite, ayant
ajouté une distance de vision à nos agent, nous avons décidé d'impacter cette
distance en fonction de la luminosité ambiante : lorsqu'il fait nuit, les
agents voient moins loin.

\subsubsection{SkyBox}
Nous avons essayé plusieurs méthodes pour le ciel : SkyBox et SkyDome. Le
skydome n'était pas adapté à notre application. Notre terrain étant une
surface carré de dimension fini il était possible voir le bas du skydome.
Or, comme le skydome consiste à appliquer une image carré sur un dôme, le
rendu sur le bas du dôme est déformé. Il est donc préférable d'utiliser une
skybox.

\subsubsection{Mesh}
Pour réaliser nos caisses, nous avons utiliser le logiciel de modélisation 3D
blender afin de réaliser le mesh. Le mesh correspond tout simplement à un
cube. Nous avons ensuite fais une texture basique avec gimp que nous avons
appliqué sur notre cube en utilisant la méthode "UV texture map".
Nous avons ensuite exporté le mesh afin de le mettre dans le format
accepté par le moteur Ogre3D.


% Fog ?
% Lights ?

\subsection{Overlays}
Pour faciliter la prise en main de notre projet nous avons décidé d'afficher
une fenêtre d'aide donnant la liste des commandes, lors de l'appui sur une
touche. Nous voulions également offrir une autre fenêtre donnant des
informations sur l'état actuel de la simulation.
Pour cela nous avons utilisé les overlays, des éléments 2D pouvant être
affiché sur la fenêtre de rendu. Ces éléments peuvent être affichés par dessus
la scène 3D et contenir du texte : ils sont donc bien adapté à ces cas
d'utilisation.

\paragraph{}
Cependant, nous avons rencontré de nombreux problèmes avec les overlays et
nous avons perdu beaucoup de temps à la résoudre. Il y a deux manières
d'utiliser les overlays, soit en écrivant le code permettant de les créer avec
divers éléments, soit en écrivant des scripts - qui sont en fait des fichiers
de descriptions des éléments de l'overlays suivant un certain format.
Afin de réduire la quantité de code et pour permettre la modification d'un
overlay dans avoir à recompiler, nous avons choisi de créer les overlays à
partir d'un script. Ce fut une mauvaise idée.

\paragraph{}
Malgré la présence d'un tutoriel simple décrivant la procédure de mise en
place d'overlay, nous avons eu du mal à obtenir le résultat voulu. En fait
même l'exemple donné par le tutoriel ne donnais pas, lorsque nous lancions
notre projet, le résultat décrit par le tutoriel. Nous observions notamment la
disparition des bordures et du texte. Après quelques investigations sur
internet, il nous est apparu que d'autres personnes avaient rencontré des
problèmes similaires sans trouver de solution.

\paragraph{}
Pour ne pas rester bloquer trop longtemps nous avons décidé d'abandonner les
scripts et de coder entièrement le format de nos overlays. Une fois de plus
nous avons rencontré des problèmes mais avons, cette fois-ci, réussi à en
comprendre la source. Le symptôme principal était le non affichage du texte,
malgré l'affichage correct des autres éléments de l'overlay. Il se trouve que
lorsqu'on créé et affiche un overlay, ce dernier restera affiché jusqu'à ce
qu'on demande à le cacher. Nous pensions donc qu'il en était de même pour le
texte contenu dans cet overlay. Il se trouve que non. Il semblerait que si
l'on ne réécrit par le texte à chaque frame, ce dernier disparaisse. Ce qui
nous semble totalement contre-intuitif étant donné que ce dernier s'affiche
par dessus des éléments de l'overlay qui, eux, restent afficher entre
plusieurs frames.

Quoiqu'il en soit, une fois la source du problème comprise, nous avons pu
le corriger et poursuivre la création de nos deux overlays.

\paragraph{Aide}
Notre première utilisation des overlays a été pour l'affichage d'une aide.
Cette aide recense les raccourcis claviers et les interactions avec la souris
que nous proposons à l'utilisateur. Cette aide décrit notamment comment bouger
la caméra, affecter l'écoulement du temps dans la simulation ou encore comment
allumer/éteindre la lumière.

\paragraph{Information}
Ce second overlay permet l'affichage de plusieurs informations au sujet de la
simulation, notamment le nombre d'agents de chaque type et le temps écoulé
depuis le lancement de la simulation.
En plus de cela, le facteur de vitesse d'écoulement du temps - détaillé par la
suite - est également affiché. Lorsque ce facteur est inférieur à 1, la
simulation est ralentie alors que quand il est supérieur à 1, la simulation
est accélérée. Nous avons fait en sorte que le temps écoulé depuis le début de
la simulation évolue plus ou moins rapidement en fonction de ce facteur.

\subsection{Interaction avec l'utilisateur}

L'utilisateur peut interagir avec l'environnement avec la souris ou au
clavier. La touche \verb!h! affiche la liste des commandes possible et la
touche i affiche des informations à propos de l'environnement.
Il est possible de déplacer la caméra dans la scène grâce à des touches
clavier et d'effectuer des rotations grâce à la souris - l'appui de la touche
\emph{shift} permet d'accélérer les translations de la caméra.

\subsubsection{Saut dans le temps et téléportation}
Pour permettre à l'utilisateur d'observer plus finement le comportement des
agents et pour voir un comportement à l'échelle macroscopique se dégager plus
rapidement, nous avons ajouté la possibilité de ralentir et d'accélérer le
temps - il est également possible d'arrêter le temps, c'est-à-dire de faire
une pause de la simulation. Dans un premier temps nous nous étions contenté de
multiplier la vitesse de base de nos agent par un facteur pour donner cette
sensation d'évolution plus lente ou plus rapide. Nous nous sommes alors aperçu
que, si cela fonctionnait bien lorsqu'on ralentissait la vitesse des agents,
lors d'une augmentation de la vitesse des agents le résultat différait de ce
que l'on attendait.

\paragraph{}
L'explication est simple, à chaque frame on calcule la nouvelle position des
agents en fonction du temps qui s'est écoulé depuis la dernière frame et de la
vitesse de l'agent. Par conséquent entre 2 frames consécutives les agents se
"téléportent" tous de leur position courante à leur nouvelle position. Lorsque
la vitesse des agents est élevée, la téléportation s'effectue sur une plus
grande distance. Le problème est que nos agents sont sensés ramasser des items
ou en déposer sous certaines conditions lorsque leur trajectoire les fait
passer près d'un item. Si la vitesse d'un agent est trop élevée, ce dernier
risque de se téléporter trop loin des items qui étaient sur sa trajectoire et
ne va donc pas les voir. On observer ainsi que lorsqu'on accélérait la vitesse
de la simulation, les agents avaient tendance à ne plus ramasser correctement
les caisses et donc on perdait l'intérêt de la démarche.

\paragraph{}
Pour palier à ce problème nous avons géré différemment l'augmentation et la
diminution de la vitesse à laquelle s'écoule le temps. Nous avons gardé la
même technique pour "ralentir" la simulation. Pour l'accélérer, nous avons
fait en sorte que, lorsque le \emph{FrameListener} appelle notre fonction de
mise à jour de la scène, plusieurs frames soient calculée au lieu d'une seule.
De cette manière les agents n'atteignent pas de vitesse trop élevée et le bon
résultat est obtenu.

% Mesh selection ?
% -> put a light on an agent
% -> kill an agent
% -> move an agent
% Ambient light

\paragraph{}Ce projet consiste en la réalisation d'une application simulant un système
multi-agent. On s'intéresse à la fois au côté comportemental et au côté réalité virtuelle.
Le but est donc de réaliser un système simpliste de type proie/prédateur.
Dans ce système, deux types d'agents sont en concurrence : Ogre et Robot. Les
ogres doivent regrouper des petites caisses en bois alors que les robots ont
pour buts d'éparpiller les caisses.

\paragraph{}Le rendu de la simulation est effectué grâce au moteur Ogre3D.
C'est un moteur open source distribué sous licence MIT. Il est compatible
avec plusieurs plateformes (Linux, Mac OSX, et Windows). Pour ce projet nous
utilisons le wrapper \CS~Mogre. Ainsi l'enjeu est l'exploration de différentes
techniques d'animation et de rendu 3D: par exemple utilisation de source de lumière ou 
encore chargement et animation d'un modèles 3D. 

\paragraph{}Nous nous sommes aussi intéressés à l'aspect comportemental. Les comportements 
des agents que nous avons implémenté, valident deux caractéristiques : vision locale et 
autonomie. D'une part, un agent doit avoir une vision partielle de son
environnement. 
Il ne peut en aucun cas réfléchir sur l'ensemble du système afin de prendre sa décision. 
D'autre part, l'agent doit prendre sa décision de manière indépendante des autres agents.
En construisant des comportements de manière locale, on s'intéresse à l'évolution globale
du système.


\paragraph{} Dans une première partie sera détaillé l'architecture de
l'application et les choix de conception réalisé. Ensuite sera expliqué le
travail effectué au niveau comportemental ainsi que les différentes
fonctionnalités graphiques réalisées. Une dernière parties décrira
l'organisation mise en place pour la réalisation du projet.

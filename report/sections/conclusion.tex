\section{Délivrable}

Une bonne partie de ensemble des fonctionnalités de bases qui étaient attendue
ont été implémentées. Nous avons notamment implémenté un suivi avec la caméra,
l'affichage d'un ciel, la gestion de lumières et de la lumière ambiante.

\section{Amélioration possibles}

Au cours de ce projet nous avons eu de nombreux idées de fonctionnalités et de
comportements à implémenter. Malheureusement, du fait des contraintes de temps
et de la charge de travail imposée par l'ensemble des projets sur lesquels
nous devions avancer en parallèle, nombre de nos idées n'ont pu émerger.

La fonctionnalité principale que nous aurions aimé avoir le temps
d'implémenter est la sélection des agents. Cela nous aurai permis de pouvoir
associer une lumière à un agent particulier en le sélectionnant à la souris ou
encore de tuer des agents à notre guise.

De plus, il aurait intéressant d'explorer des comportements plus avancés en
mettant en place un mécanisme de naissance par exemple.

\section{Conclusions personnelles}

Nous avons particulièrement apprécié ce projet du fait de la diversité des
compétences nouvelles que nous avons acquises et que nous avons pu mettre en
œuvre. Ce projet nous a notamment initié au développement sous Windows via
l'environnement Visual Studio, mais également au langage \CS qui est
particulièrement adapté à ce genre de projet qui requiert un développement
rapide et pour lequel nous apprécions que les problèmes de gestion de la
mémoire soient géré par le langage de programmation.

Nous avons également découvert (ou redécouvert) le moteur 3D Ogre3D, via le
Wrapper Mogre, nous permettant une utilisation de ce moteur 3D en \CS.

On regrettera simplement d'avoir perdu du temps sur la création des overlays,
notamment du fait d'un fonctionnement douteux des scripts.
